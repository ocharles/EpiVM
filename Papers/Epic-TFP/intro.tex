\section{Introduction}

When implementing a new language, whether for research purposes or as
a realistic general purpose language, we are inevitably faced with the
problem of executing the language. Ideally, we would like execution to
be as fast as possible, and exploit known techniques from many years
of compiler research. However, it is difficult to make use of the
existing available back ends for functional langauges, such as the
STG~\cite{evalpush,stg,llvm-haskell} or ABC~\cite{abc-machine}
machines. They may be too low level, they may make assumptions about
the source language (e.g. its type system) or there may simply be no
clearly defined API. As a result, experimental languages such as
Agda~\cite{norell-thesis} have resorted to generating Haskell, using
\texttt{unsafeCoerce} to bypass the type system. Similarly,
Cayenne~\cite{cayenne-icfp} used LML with the type checker switched
off. This is not ideal for several reasons: we cannot expect to use
the full power and optimisations of the underlying compiler, nor can
we expect it to exploit any specific features of our new source
language, such as the optimisation opportunities presented by rich
dependent type systems.

Epic aims to provide the necessary features for implementing the
back-end of a functional language --- thunks, closures, algebraic data
types, scope management, lambda lifting --- without imposing
\remph{any} design choices on the high level language designer, with
the obvious exception that a functional style is encouraged!  The
library provides \remph{compiler combinators}, which guarantee that
any output code will be syntactically correct and well-scoped.  This
gives a simple method for building a compiler for a new functional
language, e.g. for experimentation with new type systems or new domain
specific languages.

Epic was originally written as a back end for
Epigram~\cite{levitation} (the name\footnote{Coined by James McKinna}
is short for ``\textbf{Epi}gram \textbf{C}ompiler''). It is now used
by Idris~\cite{plpv11} and as an experimental back end for
Agda. It is specifically designed for reuse by other source languages.



