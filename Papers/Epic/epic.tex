\documentclass[preprint]{sigplanconf}
%\documentclass[orivec,dvips,10pt]{llncs}

\usepackage[draft]{comments}
%\usepackage[final]{comments}
% \newcommand{\comment}[2]{[#1: #2]}
\newcommand{\khcomment}[1]{\comment{KH}{#1}}
\newcommand{\ebcomment}[1]{\comment{EB}{#1}}

\usepackage{epsfig}
\usepackage{path}
\usepackage{url}
\usepackage{amsmath,amssymb} 
\usepackage{fancyvrb}

\newenvironment{template}{\sffamily}

\usepackage{graphics,epsfig}
\usepackage{stmaryrd}

\input{./macros.ltx}
\input{./library.ltx}

\NatPackage
\FinPackage

\newcounter{per}
\setcounter{per}{1}

\newcommand{\Ivor}{\textsc{Ivor}}
\newcommand{\Idris}{\textsc{Idris}}
\newcommand{\Funl}{\textsc{Funl}}
\newcommand{\Agda}{\textsc{Agda}}
\newcommand{\LamPi}{$\lambda_\Pi$}

\newcommand{\perule}[1]{\vspace*{0.1cm}\noindent
\begin{center}
\fbox{
\begin{minipage}{7.5cm}\textbf{Rule \theper:} #1\addtocounter{per}{1}
\end{minipage}}
\end{center}
\vspace*{0.1cm}
}

\newcommand{\mysubsubsection}[1]{
\noindent
\textbf{#1}
}
\newcommand{\hdecl}[1]{\texttt{#1}}

\begin{document}

\title{Epic --- a Generic Intermediate Functional Programming Language}
%\author{Edwin Brady}

\authorinfo{Edwin C. Brady}
{School of Computer Science, 
University of St Andrews, St Andrews, Scotland.}
{Email: eb@cs.st-andrews.ac.uk}


\maketitle

\begin{abstract}
Compilers for functional languages, whether strict or non-strict,
typed or untyped, need to handle many of the same problems, for
example thunks, lambda lifting, optimisation, garbage collection, and
system interaction.  Although implementation techniques are by now
well understood, it remains difficult for a new functional language to
exploit these techniques without either implementing a compiler from
scratch, or attempting fit the new language around another existing
compiler.

Epic is a compiled functional language which exposes functional
compilation techniques to a language implementor. It has both a
concrete syntax and a Haskell API. It is independent of a source
language's type system and semantics, supports eager or lazy
evaluation, and has a range of primitive types and a lightweight
foreign function interface. In this paper we describe Epic and
demonstrate its flexibility by applying it to two very different
functional languages: a dynamically typed turtle graphics language and
a dependently typed lambda calculus.

\end{abstract}

\section{Introduction}

When implementing a new language, whether for research purposes or as
a realistic general purpose language, we are inevitably faced with the
problem of executing the language. Ideally, we would like execution to
be as fast as possible, and exploit known techniques from many years
of compiler research. However, it is difficult to make use of the
existing available back ends for functional langauges, such as the
STG~\cite{evalpush,stg,llvm-haskell} or ABC~\cite{abc-machine}
machines. They may be too low level, they may make assumptions about
the source language (e.g. its type system) or there may simply be no
clearly defined API. As a result, experimental languages such as
Agda~\cite{norell-thesis} have resorted to generating Haskell, using
\texttt{unsafeCoerce} to bypass the type system. Similarly,
Cayenne~\cite{cayenne-icfp} used LML with the type checker switched
off. This is not ideal for several reasons: we cannot expect to use
the full power and optimisations of the underlying compiler, nor can
we expect it to exploit any specific features of our new source
language, such as the optimisation opportunities presented by rich
dependent type systems.

Epic aims to provide the necessary features for implementing the
back-end of a functional language --- thunks, closures, algebraic data
types, scope management, lambda lifting --- without imposing
\remph{any} design choices on the high level language designer, with
the obvious exception that a functional style is encouraged!  The
library provides \remph{compiler combinators}, which guarantee that
any output code will be syntactically correct and well-scoped.  This
gives a simple method for building a compiler for a new functional
language, e.g. for experimentation with new type systems or new domain
specific languages.

Epic was originally written as a back end for
Epigram~\cite{levitation} (the name\footnote{Coined by James McKinna}
is short for ``\textbf{Epi}gram \textbf{C}ompiler''). It is now used
by Idris~\cite{plpv11} and as an experimental back end for
Agda. It is specifically designed for reuse by other source languages.





\section{The Epic Language}

Epic is based on the untyped $\lambda$-calculus with some extensions.
It supports primitives such as strings and integers, as well as tagged
unions. There are additional control structures for specifying
evaluation order, primitive loop constructs, and calling foreign
functions. Foreign function calls are annotated with types, to assist
with marshaling values between Epic and C, but otherwise there are no
type annotations and there is no type checking --- as Epic is intended
as an intermediate language, it is assumed that the high level
language has already performed any necessary type checking. The
abstract syntax of the core language is given in Figure \ref{epicsyn}.
As a shorthand, we use de Bruijn telescope notation, $\tx$, to denote
a sequence of $\vx$.

\newcommand{\Con}[2]{\DC{Con}\:#1(#2)}

\FFIG{
\AR{
\begin{array}{rcll}\\
\vp & ::= & \vec{\VV{def}} & \mbox{(Epic program)} \\
\VV{def} & ::= & \vx(\tx) = \vt &
\mbox{(Top level definition)} \\
\\
\vt & ::= & \vx & \mbox{(Variable)} \\
& \mid &  \vt(\ttt) & \mbox{(Function application)} \\
& \mid & \lambda\vx\SC\vt & \mbox{(Lambda binding)} \\
& \mid & \RW{let}\:\vx\:=\:\vt\:\RW{in}\:\vt & \mbox{(Let
  binding)} \\
& \mid & \Con{\vi}{\ttt} & \mbox{(Constructor application)} \\
& \mid & \vt ! \vi & \mbox{(Argument projection)} \\
& \mid & \vt\:\VV{op}\:\vt & \mbox{(Infix operator)} \\
& \mid & \RW{if}\:\vt\:\RW{then}\:\vt\:\RW{else}\:\vt & \mbox{(Conditional)}\\
& \mid & \RW{case}\:\vt\:\RW{of}\:\vec{\VV{alt}} & \mbox{(Case expressions)}\\
& \mid & \RW{lazy}(\vt) & \mbox{(Lazy evaluation)} \\
& \mid & \RW{effect}(\vt) & \mbox{(Evaluate an effectful term)} \\
& \mid & \RW{while}(\vt,\vt) & \mbox{(While loops)} \\
& \mid & \vx := \:\vt\:\RW{in}\:\vt & \mbox{(Variable update)} \\
& \mid & \RW{foreign}\:\vT\:\VV{str}\:\vec{(\vt\Hab\vT)} & \mbox{(Foreign call)} \\
& \mid & \RW{malloc}\:\ve\:\ve & \mbox{(Manual allocation)} \\
& \mid & \vi \mid \vf \mid \vc \mid \vb \mid \VV{str} & \mbox{(Constants)} \\
\\
\VV{alt} & ::= &
\Con{\vi}{\tx} \cq \vt & \mbox{(Constructors)}\\
& \mid & \vi \cq \vt & \mbox{(Integer constants)} \\
& \mid & \RW{default} \cq \vt & \mbox{(Match anything)} \\
\end{array}
\medskip
\\
\begin{array}{rcll}
\VV{op} & ::= & + \mid - \mid \times \mid / \mid\:
==\: \mid \:<\: \mid \:\le\: \mid \:>\: \mid \:\ge \\
\end{array}
\medskip
\\
\begin{array}{rcll}
\vx & ::= & \mbox{Variable name} \\
\vi & ::= & \mbox{Integer literal} \\
\vf & ::= & \mbox{Floating point literal} \\
\vc & ::= & \mbox{Character literal} \\
\vb & ::= & \mbox{Boolean literal} \: \DC{True} \mid \DC{False} \\
\VV{str} & ::= & \mbox{String literal} \\
\end{array}
\medskip
\\
\begin{array}{rcll}
\vT & ::= & \TC{Int} \mid \TC{Char} \mid \TC{Bool} \mid \TC{Float}
\mid \TC{String} & \mbox{(Primitives)} \\
 & \mid & \TC{Unit} & \mbox{(Unit type)} \\
 & \mid & \TC{Ptr} & \mbox{(Foreign pointers)} \\
 & \mid & \TC{Fun} & \mbox{(Any function type)} \\
 & \mid & \TC{Data} & \mbox{(Any data type)} \\
 & \mid & \TC{Any} & \mbox{(Polymorphic type)} \\
\end{array}
}
}
{Epic syntax}
{epicsyn}

\subsection{Definitions}

An Epic program consists of a sequence of \remph{untyped} function
definitions, with zero or more arguments, 

Expressions. \texttt{let}, \texttt{case}, \texttt{lazy}

\subsection{Types}

\texttt{Int}, \texttt{Float}, \texttt{Bool}, \texttt{Data}, \texttt{Ptr},
\texttt{Unit}. Unchecked! Used for marshalling foreign functions only.

Run-time representation (31 bit ints).

\subsection{Imperative Features}

Motivation: no need to limit ourselves to functional languages, and in
some situations may help a high level language implement some optimisations.
\texttt{while}, variable update.

\subsection{Foreign Functions}

Calling, exporting.

\subsection{Haskell API}

Function names. Distinction between Haskell application and Epic
application (\texttt{@@}). Underscore convention -- it's for primitive
language constructs, arises because we can't have ``let'', ``if'',
``case'' etc as function names, and extended to other primitives such
as operators, foreign calls (anything that'd be a keyword in general).


\section{Example High Level Languages}

In this section we present compilers for three different high level
languages to demonstrate aspects of the Epic API. Firstly, we present
a compiler for the untyped $\lambda$-calculus using Higher Order
Abstract Syntax, which shows the fundamental features of Epic required
to implement a complete compiler. Secondly, we present a compiler for 
\LamPi{}~\cite{simply-easy}, a dependently typed language, which shows
how Epic can handle languages with more expressive type
systems. Finally, we present a compiler for a dynamically typed
graphics language, which shows how Epic can be used for languages with
run-time type checking and which require foreign function calls.

\subsection{Untyped $\lambda$-calculus}

\subsubsection{Representation}

Our first example is an implementation of the untyped
$\lambda$-calculus, plus primitive integers and strings, and
arithmetic and string operators. The language is represented in
Haskell using higher order abstract syntax (HOAS).  That is, we
represent $\lambda$-bindings (\texttt{Lam}) as Haskell functions,
using a Haskell variable name to refer to the locally bound
variable. We also include global reference (\texttt{Ref}) which refer
to top level functions, function application (\texttt{App}), constants
(\texttt{Const}) and binary operators (\texttt{Op}):

\begin{SaveVerbatim}{llang}

data Lang = Lam (Lang -> Lang)
          | Ref Name
          | App Lang Lang
          | Const Const
          | Op Infix Lang Lang

\end{SaveVerbatim}
\useverb{llang}

\noindent
Constants can be either integers or strings:

\begin{SaveVerbatim}{lconsts}

data Const = CInt Int
           | CStr String

\end{SaveVerbatim}
\useverb{lconsts}

\noindent
There are infix operators for arithmetic (\texttt{Plus},
\texttt{Minus}, \texttt{Times} and \texttt{Divide}), string
manipulation (\texttt{Append}) and comparison (\texttt{Eq},
\texttt{Lt} and \texttt{Gt}). The comparison operators return an
integer --- zero if the comparison is true, non-zero otherwise:

\begin{SaveVerbatim}{lops}

data Infix = Plus | Minus | Times | Divide | Append
           | Eq   | Lt    | Gt

\end{SaveVerbatim}
\useverb{lops}

\noindent
A complete program consists of a collection of named \texttt{Lang}
definitions:

\begin{SaveVerbatim}{lprogs}

type Defs = [(Name, Lang)]

\end{SaveVerbatim}
\useverb{lprogs}

\subsubsection{Compilation}

Our aim is to convert a collection of \texttt{Defs} into an
executable, using one of the following functions from the Epic API:

\useverb{compepic}

\noindent
Given an Epic \texttt{Program}, \texttt{compile} will generate an
executable, and \texttt{run} will generate an executable then run it.
Recall that a program is a collection of named Epic declarations:

\begin{SaveVerbatim}{eprogs}

data EpicDecl = forall e. EpicExpr e => EpicFn Name e
              | ...

type Program = [EpicDecl]

\end{SaveVerbatim}
\useverb{eprogs}

Our goal, then, is to convert a \texttt{Lang} definition into
something which is an instance of \texttt{EpicExpr}. We use
\texttt{Term}, which is an Epic expression which carries a name
supply. Most of the term construction functions in the Epic API return
a \texttt{Term}.

\begin{SaveVerbatim}{buildtype}

build :: Lang -> Term

\end{SaveVerbatim}
\useverb{buildtype}

\noindent
The full implementation of \texttt{build} is given in Figure \ref{lcompile}.
In general, this is a straightforward traversal of the \texttt{Lang}
program, converting \texttt{Lang} constants to Epic constants,
\texttt{Lang} application to Epic application, and \texttt{Lang}
operators to the appropriate built-in Epic operators. 
                  
\begin{SaveVerbatim}{lcompile}

build :: Lang -> Term
build (Lam f)          = term (\x -> build (f (EpicRef x)))
build (EpicRef x)      = term x
build (Ref n)          = ref n
build (App f a)        = build f @@ build a
build (Const (CInt x)) = int x
build (Const (CStr x)) = str x
build (Op Append l r)  = fn "append" @@ build l @@ build r
build (Op op l r)      = op_ (eOp op) (build l) (build r)
    where eOp Plus   = plus_
          eOp Minus  = minus_
          eOp Times  = times_
          eOp Divide = divide_
          eOp Eq     = eq_
          eOp Lt     = lt_
          eOp Gt     = gt_

\end{SaveVerbatim}
\codefig{lcompile}{Compiling Untyped $\lambda$-calculus}

The cases worth noting are the compilation of $\lambda$-bindings and
string concatenation. Using HOAS has the advantage that Haskell can
manage scoping, but the disadvantage that it is not straightforward to
convert the abstract syntax into another form. The Epic API also
allows scope management using HOAS, so we need to convert a function
where the bound name refers to a \texttt{Lang} value into a function
where the bound name refers to an Epic value. The easiest solution is
to extend the \texttt{Lang} datatype with an Epic reference:

\begin{SaveVerbatim}{lextend}

data Lang = ...
          | EpicRef Expr

build (Lam f) = term (\x -> build (f (EpicRef x)))

\end{SaveVerbatim}
\useverb{lextend}

\noindent
To convert a \texttt{Lang} function to an Epic function, we build an
Epic function in which we apply the \texttt{Lang} function to the Epic
reference for its argument. Every reference to a name in \texttt{Lang}
is converted to the equivalent reference to the name in Epic. While
there may be neater solutions involving an environment, or avoiding
HOAS, this solution is very simple to implement, and preserves the
desirable feature that Haskell manages scope.

Compiling string append uses a built in function provided by the Epic
interface in \texttt{basic\_defs}:

\begin{SaveVerbatim}{lappend}

build (Op Append l r) 
       = fn "append" @@ build l @@ build r

\end{SaveVerbatim}
\useverb{lappend}

\noindent
Given \texttt{build}, we can translate a collection of HOAS
definitions into an Epic program, add the built-in Epic definitions
and execute it directly. Recall that there must be a function called
\texttt{"main"} or Epic will report an error.

\begin{SaveVerbatim}{lmain}

mkProgram :: Defs -> Program
mkProgram ds = basic_defs ++ 
               map (\ (n, d) -> EpicFn n (build d)) ds

execute :: Defs -> IO ()
execute p = run (mkProgram p)

\end{SaveVerbatim}
\useverb{lmain}

\noindent
Alternatively, we can generate an executable. Again, the entry point
is the Epic function called \texttt{"main"}:

\begin{SaveVerbatim}{lcomp}

comp :: Defs -> IO ()
comp p = compile "a.out" (mkProgram p)

\end{SaveVerbatim}
\useverb{lcomp}

\noindent
This is a compiler for a very simple language, but a compiler for any
more complex language using the Epic API follows the same pattern:
convert the abstract syntax for each named definition into a named Epic
\texttt{Term}, add any required primitives (we have just used
\texttt{basic\_defs} here), and pass the collection of definitions to
\texttt{run} or \texttt{compile}.

\subsection{Dependently Typed $\lambda$-calculus}

\LamPi{}~\cite{simply-easy}. Complications: elimination
operators. Representation uses de Bruijn indices. Need a way to dump
output. Non-complication: odd type system.

\subsubsection{Representation}

\subsubsection{Compilation}

\subsection{A Dynamically Typed Turtle Graphics Language}





\section{Implementation}

How it's implemented is not really important to the user --- a
compiler can target Epic without knowing, and we could drop in a new
back end at any time in principle.

There is currently one back end, but more are planned. Compiled via
C. Garbage collection with Boehm~\cite{boehm-gc},
\texttt{\%memory}. (Note that a non-moving collector makes things
easier for foreign functions, but may not be the best choice in the
long run).

Later plans: compile via LLVM, allow plug in garbage collectors
(important for embedded systems, device drivers, operating system
services, for example).



\input{performance}

\section{Related Work}

%% GHC's run-time system~\cite{stg, evalpush}, ABC
%% machine~\cite{abc-machine} and why we don't just use one of them
%% (no useful interface, imposes constraints on the type system).
%% Some influence from GRIN~\cite{grin-project}.

Epic is currently used by Agda and Idris~\cite{plpv11}, as well as the
development version of Epigram~\cite{levitation}. Initial
benchmarking~\cite{scrap-engine} shows that the code generated by Epic
can be competitive with Java and is not significantly worse than C.
Epic uses techniques from other functional language back
ends~\cite{evalpush,stg,abc-machine} but deliberately exposes its core
language as an API to make it as reusable as possible. Although there
is likely to always be a trade off between reusability and efficiency,
exposing the API will make it easier for other language researchers to
build a new compiler quickly. The Lazy Virtual Machine~\cite{lvm} has
similar goals but it designed as a lower level target language, rather
than a high level API. 

%C--~\cite{c--} and LLVM~\cite{llvm}
%as possible code generation strategies. Supercompilation for
%optimisations~\cite{mitchell-super}.

\section{Conclusion}

Epic provides a simple path for language researchers to convert
experimental languages (e.g. experimenting with new type systems or
domain specific language design) into larger scale, usable tools, by
providing an API for generating a compiler, dealing with
well-understood but difficult to implement problems such as lambda
lifting, code generation, interfacing with foreign functions and
garbage collection.



%\vspace{-0.2in}
%% \section*{Acknowledgments}

%% This work was partly funded by the Scottish Informatics and Computer
%% Science Alliance (SICSA) and by EU Framework 7 Project No. 248828
%% (ADVANCE). I thanks James McKinna, Kevin Hammond and Anil
%% Madhavapeddy for several helpful discussions, and the anonymous
%% reviewers for their constructive suggestions.

\bibliographystyle{abbrv}
\begin{small}
\bibliography{literature.bib}

\appendix

%\input{code}

\end{small}
\end{document}
